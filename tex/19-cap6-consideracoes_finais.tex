\chapter{Considerações finais}

Este capítulo descreve as observações posteriores à conclusão do projeto.

\section{Conclusões do projeto de formatura}
% "foi bom, tivemos que mudar isso e isso e isso"
Neste projeto foram explorados diversos tópicos relacionados ao desenvolvimento de um sistema de compilação de dados, abrangendo desde os requisitos funcionais até a implementação prática de tecnologias e ferramentas específicas. Foram discutidos aspectos cruciais, como segurança da informação, integração de hardware e análise de dados. A abordagem envolveu desde a coleta de dados por meio de sensores até a visualização interativa em mapas. 

A integração de tecnologias como AWS RDS, MySQL, OBD-II, Python e Folium destacou a diversidade de ferramentas disponíveis para a construção de sistemas robustos de captura de dados. Este projeto enfatizou a privacidade dos dados e a  observação de informações essenciais, oferecendo insights valiosos sobre o desempenho, eficiência e manutenção dos veículos. Dessa forma, a concepção e implementação de uma plataforma de captura de dados exigiu uma abordagem multidisciplinar, considerando não apenas a tecnologia, mas também os aspectos éticos e práticos para oferecer uma solução eficaz, segura e alinhada às necessidades do usuário.

\section{Contribuições}
% qual foi a contrib da equipe

\section{Dificuldades encontradas e lições aprendidas}

\subsection{Salvamento de dados}
    
    As dificuldades encontradas ao salvar arquivos no celular Android e na nuvem da AWS podem ser atribuídas à falta de clareza na documentação desses serviços.
    
    A ausência de informações completas e operacionais resultou em desafios significativos durante a implementação deste projeto.
    
    Dessa forma, o grupo começou a valorizar a busca constante por códigos chamados \textit{hello world}, os quais provam um conceito que deve ser parte de uma solução mais complexa.
    
    O maior aprendizado deste projeto é a definição de passos o mais atômicos possíveis para que o projeto possa caminhar em solo firme do começo ao fim, isto é, todos os submódulos do projeto devem ter um plano de testes claro para que os engenheiros que lidarão com ele no futuro sejam capazes de solucionar problemas o mais rápido possível.

\subscetion{Tecnologias desconhecidas}
    O grupo chegou à conclusão que a especificação das tecnologias utilizadas para um projeto devem sempre levar em conta o conhecimento e a capacidade da equipe envolvida no desenvolvimento.

    Embora sempre haja novos conhecimentos a serem explorados, é importante limitar a quantidade de dúvidas inciais do projeto e saná-las o mais rápido possível.

    Essa abordagem não foi adotada da melhor forma neste trabalho, o que gerou muito \textit{stress} para a geração de código, principalmente durante o desenvolvimento do aplicativo \textit{Android}.

\section{Perspectivas de continuidade}
% o que pode ser feito depois do tcc?
A oferta deste sistema de coleta de dados do veículos para empresas de seguro pode representar uma proposta de valor significativa. Ao utilizar uma solução que integra tecnologias avançadas de análise de dados, as seguradoras podem aprimorar suas avaliações de risco de maneira eficiente. 

O sistema oferece a capacidade de monitorar o comportamento de condução em tempo real, identificar padrões de condução segura e avaliar incidentes, permitindo uma avaliação mais precisa e personalizada do risco associado a cada segurado. 

Além disso, com as métricas recolhidas, é possível fornecer feedback direto aos motoristas ou oferecer descontos e incentivos para práticas de condução seguras. O histórico de rotas e a telemetria abrangente proporciona transparência e eficiência no processo de precificar o seguro. 

Ao posicionar esse sistema como uma ferramenta inovadora para melhorar a gestão de riscos, as empresas de seguro podem não apenas otimizar seus processos, mas também criar um diferencial competitivo valioso, fortalecendo a fidelidade do cliente e promovendo a segurança viária de maneira colaborativa.