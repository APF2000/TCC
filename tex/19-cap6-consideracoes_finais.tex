\chapter{Considerações finais}

Este capítulo descreve as observações posteriores à conclusão do projeto.

\section{Conclusões do projeto de formatura}
% "foi bom, tivemos que mudar isso e isso e isso"
Neste projeto foram explorados diversos tópicos relacionados ao desenvolvimento de um sistema de compilação de dados, abrangendo desde os requisitos funcionais até a implementação prática de tecnologias e ferramentas específicas. Foram discutidos aspectos cruciais, como segurança da informação, integração de hardware e análise de dados. A abordagem envolveu desde a coleta de dados por meio de sensores até a visualização interativa em mapas. 

A integração de tecnologias como AWS RDS, MySQL, OBD-II, Python e Folium destacou a diversidade de ferramentas disponíveis para a construção de sistemas robustos de captura de dados. Este projeto enfatizou a privacidade dos dados e a  observação de informações essenciais, oferecendo insights valiosos sobre o desempenho, eficiência e manutenção dos veículos. Dessa forma, a concepção e implementação de uma plataforma de captura de dados exigiu uma abordagem multidisciplinar, considerando não apenas a tecnologia, mas também os aspectos éticos e práticos para oferecer uma solução eficaz, segura e alinhada às necessidades do usuário.

\section{Contribuições}
% qual foi a contrib da equipe
As contribuições de Guilherme Migliatti foram essenciais para superar obstáculos no desenvolvimento do aplicativo, oferecendo suporte técnico, que permitiu a compilação do app. Sua dedicação foi fundamental para garantir a funcionalidade do sistema de compilação de dados.

Além disso, a contribuição significativa do Sr. Jarrdin, que disponibilizou o simulador de OBD, ampliou consideravelmente as capacidades de teste do sistema. 

O acesso ao simulador permitiu uma abordagem mais abrangente e controlada durante as fases de desenvolvimento, facilitando a simulação de diferentes cenários e otimizando a integração do OBD-II ao sistema de coleta de dados. As contribuições, de Guimigli e Sr. Jarrdin, desempenharam papéis importantes no progresso do projeto, demonstrando a importância da colaboração e apoio na comunidade de desenvolvimento.

\section{Perspectivas de continuidade}
% o que pode ser feito depois do tcc?
A oferta deste sistema de coleta de dados do veículos para empresas de seguro pode representar uma proposta de valor significativa. Ao utilizar uma solução que integra tecnologias avançadas de rastreamento e análise de dados, as seguradoras podem aprimorar suas avaliações de risco de maneira eficiente. 

O sistema oferece a capacidade de monitorar o comportamento de condução em tempo real, identificar padrões de condução segura e avaliar incidentes, permitindo uma avaliação mais precisa e personalizada do risco associado a cada segurado. 

Além disso, com as métricas recolhidas, é possível fornecer feedback direto aos motoristas ou oferecer descontos e incentivos para práticas de condução seguras. O histórico de rotas e a telemetria abrangente proporciona transparência e eficiência no processo de precificar o seguro. 

Ao posicionar esse sistema como uma ferramenta inovadora para melhorar a gestão de riscos, as empresas de seguro podem não apenas otimizar seus processos, mas também criar um diferencial competitivo valioso, fortalecendo a fidelidade do cliente e promovendo a segurança viária de maneira colaborativa.