\chapter{Considerações finais}

Este capítulo descreve as observações posteriores à conclusão do projeto.

\section{Conclusões do projeto de formatura}
% "foi bom, tivemos que mudar isso e isso e isso"
Esta seção será modificada quando o projeto tiver sido implementado.

\section{Contribuições}
% qual foi a contrib da equipe
Esta seção será modificada quando o projeto tiver sido implementado.

\section{Perspectivas de continuidade}
% o que pode ser feito depois do tcc?
A partir do sistema que será desenvolvido é possível vislumbrar algumas aplicações que podem ser desenvolvidas após o fim do trabalho.

A ideia em todos os casos a seguir seria fazer uso da plataforma de dados como ferramenta principal de fornecimento de informações.

\begin{itemize}
    \item \textbf{Classificador de perfil de direção para seguros de carro personalizados:} seria uma forma de precificar o seguro de cada pessoa, dependendo de seu perfil de condução.
    
    \item \textbf{Prova de direção conduzida por Inteligência Artificial:} possível supressão da prova do Detran, avaliando o condutor ao longo das aulas práticas.
    
    \item \textbf{Ranqueador de segurança de carros autônomos:} avaliação dos carros do futuro para determinar qual dirige com mais segurança.
    
    \item \textbf{Revisão remota / diagnóstico automático de carro:} relatar problemas antes que aconteçam, sem a necessidade de um mecânico para examinar diretamente o veículo.
    
    \item \textbf{Classificação de motoristas de aplicativo:} seria uma forma de avaliação complementar à fornecida pelos usuários de plataformas como \textit{Uber} e \textit{99 Taxi}, adicionando o elemento de segurança de direção.
\end{itemize}