\chapter{Aspectos conceituais}
\label{CAP2}

A seguir serão explicados os conceitos fundamentais que possibilitam a execução deste trabalho. 

% apresentar conceitos empregados e revisao da literatura (parte toerica do trab)

\section{O padrão OBD-II}

Criado na década de 1990 para gerar controle sobre as emissões de gás carbônico dos carros [REF X] \cite{patent}, hoje define um protocolo padronizado para comunicar parâmetros internos do veículo.
    
O padrão OBD-II foi uma extensão do padrão OBD-I, uniformizando esse tipo de conector em casos em geral, começando a ser adotado no Brasil a partir de 2010[REF 4].

A comunicação é feita através de uma conexão física que normalmente pode ser encontrada abaixo do volante do motorista [REF 5].

Nesse padrão de conexão e comunicação é definida uma interface por onde parâmetros internos a um carro podem ser monitorados.

As mensagens definidas pelo OBD-II têm cada uma um PID (parameter ID). O conjunto comum de PIDs de serviços que podem ser solicitados pela porta OBD-II e para que servem pode ser visto na tabela a seguir[REF 3].

\begin{table}[]
% \begin{adjustbox}{width=\textwidth}
\begin{tabular}{cc}
\rowcolor[HTML]{656565} 
{\color[HTML]{FFFFFF} Service / Mode (hex)} & {\color[HTML]{FFFFFF} Description}                                                                    \\
01                                          & Show current data - I/M Monitors and Live Data                                                        \\
02                                          & Show Freeze Frame (FF) Data                                                                           \\
03                                          & Show Stored Diagnostic Trouble Codes                                                                  \\
04                                          & Clear/Erase Diagnostic Trouble Codes and stored values                                                \\
05                                          & Test results, oxygen sensor monitoring (non CAN only)                                                 \\
06                                           & Test results, other component/system monitoring (Test results, oxygen sensor monitoring for CAN only) \\
07                                           & Show pending Diagnostic Trouble Codes (detected during current or last driving cycle)                 \\
08                                           & Control operation of on-board component/system (EVAP)                                                 \\
09                                          & Request Vehicle Information (VIN)                                                                     \\
0A                                          & Permanent Diagnostic Trouble Codes (DTCs) (Cleared DTCs)                                             
\end{tabular}
\end{table}

É importante notar que os serviços do padrão OBD-II comuns a todos os carros sempre começam com o dígito zero (hexadecimal), por isso, serviços específicos de cada fabricante devem começam a partir do código 0x10.

\begin{figure}[hp]
    \centering
    
    \includegraphics[]{figures/localizacao_obd2.png}
    
    \caption{Posição da porta OBD-II em um carro[REF 5]}
\end{figure}

Os dados que podem ser coletados de qualquer carro são explicitados na tabela abaixo[REF 3]:

\begin{table}[h]
% \begin{adjustbox}{width=\textwidth}
    \caption{PIDs do OBD-II e seus valores de referência.}
    \label{Tb:tab2}
    \centering
    
    \begin{adjustbox}{width=\textwidth}
    
        \begin{tabular}{ccc p{6cm} ccc}
            \rowcolor[HTML]{C0C0C0} 
            
            {\color[HTML]{FFFFFF} \textbf{PIDs(hex)}} & {\color[HTML]{FFFFFF} \textbf{PID(Dec)}} & {\color[HTML]{FFFFFF} \textbf{Data bytes returned}} & {\color[HTML]{FFFFFF} \textbf{Description}} & {\color[HTML]{FFFFFF} \textbf{Min value}} & {\color[HTML]{FFFFFF} \textbf{Max value}} & {\color[HTML]{FFFFFF} \textbf{Units}} \\
            
            04 & 4 & 1           & Calculated engine load                      & 0 & 100 & \% \\
            0C & 12 & 2           & Engine speed & 0 & 16,383.75 & rpm \\
            0D & 13 & 1           & Vehicle speed & 0 & 255 & km/h \\
            0F & 15 & 1           & Intake air temperature                      & -40 & 215 & °C \\
            1F & 31 & 2           & Run time since engine start                 & 0 & 65,535 & seconds \\
            33 & 51 & 1           & Absolute Barometric Pressure                & 0 & 255 & kPa \\
            46 & 70 & 1           & Ambient air temperature                     & -40 & 215 & °C \\
            47 & 71 & 1           & Absolute throttle position B                & 0 & 100 & \% \\
            49 & 73 & 1           & Accelerator pedal position D                & 0 & 100 & \% \\
            51 & 81 & 1           & Fuel Type & - & - & - \\
            52 & 82 & 1           & Ethanol fuel \% & 0 & 100 & \% \\
            5A & 90 & 1           & Relative accelerator pedal position         & 0 & 100 & \% \\
            5B & 91 & 1           & Hybrid battery pack remaining life          & 0 & 100 & \% \\
            5C & 92 & 1           & Engine oil temperature                      & -40 & 210 & °C \\
            5D & 93 & 2           & Fuel injection timing & -210.00 & 301992 & ° \\
            5E & 94 & 2           & Engine fuel rate & 0 & 3212.75 & L/h \\
            61 & 97 & 1           & Driver's demand engine - percent torque     & -125 & 130 & \% \\
            62 & 98 & 1           & Actual engine - percent torque              & -125 & 130 & \% \\
            63 & 99 & 2           & Engine reference torque                     & 0 & 65,535 & Nm \\
            64 & 100 & 5           & Engine percent torque data                  & -125 & 130 & \% \\
            70 & 112 & 10          & Boost pressure control                      & - & - & - \\
            83 & 131 & 9           & NOx sensor & - & - & - \\
            8E & 142 & 1           & Engine Friction - Percent Torque            & -125 & 130 & \
            %     https://pt.overleaf.com/project/625c13a8f632581c97970222 
        \end{tabular}
    \end{adjustbox}
\end{table}

% \noindent

% \begin{tblr}[caption = {Modelos estatísticos e suas relações.}]{
%       hlines,
%       vlines,
%       row{1} = {bg=gray,fg=white},
%       rows = {halign=c},
%       columns = {halign=c},
%     %   column{4} = {Q[3cm]},
%     %   column{2} = {Q[1cm] XXX}
%       colspec = {Q[1.2cm, halign=c] XXX },
%   } 
%     \label{Tb:tab1}
%     \centering
  
%   \rowcolor[HTML]{C0C0C0} 
            
%             {\color[HTML]{FFFFFF} \textbf{PIDs (hex)}} & {\color[HTML]{FFFFFF} \textbf{PID (Dec)}} & {\color[HTML]{FFFFFF} \textbf{Data bytes returned}} & {\color[HTML]{FFFFFF} \textbf{Description}} & {\color[HTML]{FFFFFF} \textbf{Min value}} & {\color[HTML]{FFFFFF} \textbf{Max value}} & {\color[HTML]{FFFFFF} \textbf{Units}} \\
  
%               04 & 4 & 1           & Calculated engine load                      & 0 & 100 & \% \\
%             0C & 12 & 2           & Engine speed & 0 & 16,383.75 & rpm \\
%             0D & 13 & 1           & Vehicle speed & 0 & 255 & km/h \\
%             0F & 15 & 1           & Intake air temperature                      & -40 & 215 & °C \\
%             1F & 31 & 2           & Run time since engine start                 & 0 & 65,535 & seconds \\
%             33 & 51 & 1           & Absolute Barometric Pressure                & 0 & 255 & kPa \\
%             46 & 70 & 1           & Ambient air temperature                     & -40 & 215 & °C \\
%             47 & 71 & 1           & Absolute throttle position B                & 0 & 100 & \% \\
%             49 & 73 & 1           & Accelerator pedal position D                & 0 & 100 & \% \\
%             51 & 81 & 1           & Fuel Type & - & - & - \\
%             52 & 82 & 1           & Ethanol fuel \% & 0 & 100 & \% \\
%             5A & 90 & 1           & Relative accelerator pedal position         & 0 & 100 & \% \\
%             5B & 91 & 1           & Hybrid battery pack remaining life          & 0 & 100 & \% \\
%             5C & 92 & 1           & Engine oil temperature                      & -40 & 210 & °C \\
%             5D & 93 & 2           & Fuel injection timing & -210.00 & 301992 & ° \\
%             5E & 94 & 2           & Engine fuel rate & 0 & 3212.75 & L/h \\
%             61 & 97 & 1           & Driver's demand engine - percent torque     & -125 & 130 & \% \\
%             62 & 98 & 1           & Actual engine - percent torque              & -125 & 130 & \% \\
%             63 & 99 & 2           & Engine reference torque                     & 0 & 65,535 & Nm \\
%             64 & 100 & 5           & Engine percent torque data                  & -125 & 130 & \% \\
%             70 & 112 & 10          & Boost pressure control                      & - & - & - \\
%             83 & 131 & 9           & NOx sensor & - & - & - \\
%             8E & 142 & 1           & Engine Friction - Percent Torque            & -125 & 130 & \
% \end{tblr}

Interessante notar que, embora não esteja representado na tabela, os PIDs 0x00, 0x20, 0x40, etc, ou seja, a cada 32 valores de PID, existe um parâmetro apenas para indicar quais entre os PIDs seguintes são fornecidos por aquele carro.

Esses PIDs de marcação, por assim dizer, utilizam 4 bytes para comunicar quais dos próximos PIDs estão disponíveis, utilizando cada bit como uma flag binária. 

Na imagem a seguir é possível visualizar como esse processo é feito, usando como exemplo o valor 0xBE1FA813 para representar os 4 bytes oferecidos[REF 3].

\begin{figure}[hp]
    \centering
    
    \includegraphics[scale=0.7]{figures/tabela_dados_disponiveis.png}
    
    \caption{}
\end{figure}

A coleta de fato dos dados relevantes será feita por uma aplicação desenvolvida \textit{a priori} que já é capaz de comunicar-se corretamente com a interface OBD.

Essa aplicação acelerará a criação da infraestrutura proposta por este trabalho, uma vez que pula o primeiro passo da estratégia \textit{bottom-up} que deverá permear o projeto.

\section{Transmissão de dados}
Será feita por um dispositivo auxiliar (celular ou placa de arduino) utilizando algum protocolo de comunicação sem fio com a internet.

A princípio, pensa-se em usar o 4G do celular como meio de transporte das informações coletadas pela porta OBD-II, para simplificar o caminho até o primeiro MVP.

\section{Armazenamento em nuvem}
Provedoras como a Amazon e a Azure (Microft) oferecem serviços de nuvem que podem ser usados para este projeto [REF 10 e 11].

Os usuários poderão transmitir informações para a nuvem a qualquer momento, desde que seus carros estejam ligados e conectados à internet.

O volume de dados recebidos no sistema é, portanto, variável e por isso faz sentido que o serviço de nuvem a ser contratado deverá seguir algum modelo \textit{on demand}.