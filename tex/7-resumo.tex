\chapter*{Resumo}

Com o objetivo de criar uma plataforma complementar ao motorista e seu veículo, este projeto aproveitou a infraestrutura presente em carros e \textit{smartphones} para ser realizado.

A coleta de informações do veículo foi realizada por meio da porta OBD-II, presente em todos os veículos fabricados a partir de 2010 no Brasil, enquanto dados do celular incluíram informações cruciais sobre a movimentação do veículo, como geolocalização e aceleração.

A subsequente transferência desses dados para a nuvem da AWS viabilizou a análise do perfil de cada condutor. Essa abordagem não apenas proporciona uma compilação de dados provenientes de diferentes fontes, mas também permite uma avaliação aprofundada das preferências e comportamentos individuais de condução, contribuindo para uma compreensão mais completa do desempenho veicular e dos padrões de uso.

