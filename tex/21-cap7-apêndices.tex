\chapter{Apêndices}

% Como juntar todas as subpartes do projeto é uma outra tarefaimportante de engenharia.

% A seguir é descrito como isso pode ser feito.

%%%%%%%%%%%%%%%%%%%

\section{Conversa com um profissional da Porto Seguro sobre seguros de carro}

%%%%%%%%%%%%%%%%%%%

\textbf{Pergunta 1:} 
Quanto aos locais por onde o motorista passa ao longo do dia, é mais importante saber o nome do bairro por onde ele passa ou se soubermos a rua é melhor pra sabermos o risco de assalto?

\textbf{Resposta 1:} 
A precificação do seguro já é baseada em três fatores: onde a pessoa mora, onde ela trabalha e o trajeto feito por ela para ir de um local ao outro.
Na nossa plataforma, chamada \textit{Corretor Online}, temos acesso a esses e outro dados do cliente, o que nos possibilita precificar o seguro de forma personalizada.

%%%%%%%%%%%%%%%%%%%

\textbf{Pergunta 2:} 
É importante a hora em que a pessoa passa em cada lugar? Ou dia da semana é um fator mais determinístico?

\textbf{Resposta 2:}
Não há correlação provada entre o dia da semana em que a pessoa dirige e a probabilidade de acidente de carro.

A hora do dia, no entanto, como pode ser intuitivo para muitos, determina a chance de o carro ser roubado. A maior parte dos crimes ocorre à noite.

%%%%%%%%%%%%%%%%%%%

% \textbf{Pergunta 3:} Momentos com mais trânsito podem ser um fator de risco para acidentes?


%%%%%%%%%%%%%%%%%%%

\textbf{Pergunta 3:} O sistema de precificação por uso é interessante pra seguros de motoristas de aplicativos? Quais parâmetros são mais relevantes para quem usa o carro prolongadamente?

\textbf{Resposta 3:}
Sim, com certeza seria interessante um sistema desse tipo.

Sobre os parâmetros exatos que seriam mais interessantes, preciso consultar a área estatística da empresa para ter mais certeza, mas certamente informações sobre como o motorista usa o carro (a quantidade de freadas bruscas ao longo do trajeto, por exemplo) e por onde passa com ele, conforme mencionei anteriormente, são fundamentais, na minha experiência.

%%%%%%%%%%%%%%%%%%%

\textbf{Pergunta 4:} 
Empresas podem se beneficiar da redução do seguro também? Levando-se em conta viagens feitas por funcionários, a trabalho

\textbf{Resposta 4:}
Hoje em dia nós seguramos empresas baseado puramente em quanto veículos têm na frota delas e quanto eles serão usados em média.

Se fosse possível diferenciar um motorista do outro, o cálculo do risco seria mais preciso, além de poder incentivar a boa condução.

Fizemos uma campanha com ideia parecida uma vez, chamada \textit{Trânsito Mais Gentil}. A ideia era diminuir o prêmio cobrado de cada segurado caso ele comprovasse, ao fim do ano, que cometeu poucas ou nenhuma infração no trânsito, dando-nos a quantidade de pontos recebidos na carteira.

%%%%%%%%%%%%%%%%%%%
    
\textbf{Pergunta 5:} 
Existe o interesse de ser feita uma precificação mais personalizada de seguros de carro, usando-se dados de estilo de direção de cada indivíduo em vez de estatísticas gerais?

\textbf{Resposta 5:}
Definitivamente sim, e acredito que essa seja a tendência do mercado no futuro próximo.

Existe uma plataforma que implementa algo parecido com o que vocês estão desenvolvendo, chama-se Azos\textsuperscript{[30]}.

Essa empresa precifica seguros de vida de forma dinâmica, levando em consideração não apenas estatísticas gerais sobre doenças que cada pessoa pode desenvolver, mas também e principalmente os hábitos do dia a dia de cada um e quais as novas condições de saúde do segurado a cada renovação de contrato.

%%%%%%%%%%%%%%%%%%%
    
\textbf{Pergunta 6:} 
Alguma consideração final?
    
\textbf{Resposta 6:} 
Quanto mais ajustado ao verdadeiro risco, melhor será o produto tanto para o cliente como para a seguradora, pois o preço mais justo para aquele serviço será o efetivamente cobrado.

Dito isso, vejo com muito bons olhos o projeto de vocês e contribuo com uma sugestão: algo tão personalizado assim poderia implementar um sistema equivalente ao \textit{Open Banking}, de que tanto se fala hoje em dia, uma vez que o histórico de cada cliente não é transferível entre empresas de seguro.

%%%%%%%%%%%%%%%%%%%


\section{Colunas da tabela do banco de dados}

\begin{python}
CREATE TABLE `tcc_main`.`info` (
  `id` INT NOT NULL AUTO_INCREMENT, 
  `timestamp` TIMESTAMP(3) NOT NULL,
  `user_token` VARCHAR(40),
  `name` VARCHAR(45) NOT NULL,
  `result` VARCHAR(45) NOT NULL,
  PRIMARY KEY (`id`),
  UNIQUE INDEX `id_UNIQUE` (`id` ASC) VISIBLE);

CREATE TABLE `acceleration` (
  `id` INT NOT NULL AUTO_INCREMENT,
  `timestamp` TIMESTAMP(3) NOT NULL,
  `user_token` VARCHAR(40),
  `acceleration_x` VARCHAR(10) NOT NULL,
  `acceleration_y` VARCHAR(10) NOT NULL,
  `acceleration_z` VARCHAR(10) NOT NULL,
  `gravity_x` VARCHAR(10) NOT NULL,
  `gravity_y` VARCHAR(10) NOT NULL,
  `gravity_z` VARCHAR(10) NOT NULL,
  PRIMARY KEY (`id`),
  UNIQUE INDEX `id_UNIQUE` (`id` ASC) VISIBLE);

CREATE TABLE `location` (
  `id` INT NOT NULL AUTO_INCREMENT,
  `timestamp` TIMESTAMP(3) NOT NULL,
  `user_token` VARCHAR(40),
  `latitude` VARCHAR(20) NOT NULL,
  `longitude` VARCHAR(20) NOT NULL,
  PRIMARY KEY (`id`),
  UNIQUE INDEX `id_UNIQUE` (`id` ASC) VISIBLE);

\end{python}


%%%%%%%%%%%%%%%%%%%%%%%%%%%%%%%



\section{Código para o Lambda da AWS de comunicação com o banco de dados}

\begin{python}

import awsgi
import pymysql
from flask import Flask, request

app = Flask(__name__)

@app.route("/app_data", methods=["POST"])
def api_data():
    host = "open-db.clznpmxaqrkl.us-east-1.rds.amazonaws.com"
    user = "testadm"
    password = "testadm#1"
    database = "tcc_schema"
    
    #Connection
    connection = pymysql.connect(host=host, user=user, 
    password=password, database=database)
    cursor = connection.cursor()

    data = request.get_json()
    
    method_name = data["method"]

    if method_name.split("_")[0] == "get" :
        filters = {
            "time_min" : data["time_min"],
            "time_max" : data["time_max"],
            "user_token" : data["user_token"]
        }
        
    data = data.get("data", [])

    if(method_name == "add_obd_info"):
        print("inserto data to info")

        query_insert_cmd = "INSERT INTO info (timestamp, user_token, name, 
        result) VALUES (\%(timestamp)s, \%(user_token)s, \%(name)s, \%(result)s)"
        cursor.executemany(query_insert_cmd, data)

        connection.commit()
        return "deu bom"
    
    elif(method_name == "add_acceleration"):
        print("insert data to acceleration")

        query_insert_cmd = "INSERT INTO acceleration (timestamp, user_token, 
        acceleration_x, acceleration_y, acceleration_z, gravity_x, gravity_y, 
        gravity_z) VALUES (\%(timestamp)s, \%(user_token)s, \%(acceleration_x)s,
        \%(acceleration_y)s, \%(acceleration_z)s, \%(gravity_x)s, \%(gravity_y)s,
        \%(gravity_z)s)"
        cursor.executemany(query_insert_cmd, data)

        connection.commit()
        return "deu bom"
    
    elif(method_name == "add_location"):
        print("insert data to location")

        query_insert_cmd = "INSERT INTO location (timestamp, user_token,
            latitude, longitude) VALUES (\%(timestamp)s, \%(user_token)s, 
            \%(latitude)s, \%(longitude)s)"
        cursor.executemany(query_insert_cmd, data)

        connection.commit()
        return "deu bom"
    
    elif(method_name == "get_obd_info"):
        print("get data from info")
        
        query_select_cmd = "SELECT * FROM info where timestamp >= 
            '\%(time_min)s' and timestamp <= '\%(time_max)s' 
            and user_token = '\%(user_token)s'" \%filters

        cursor.execute(query_select_cmd)

        return {"data": cursor.fetchall()}
    
    elif(method_name == "get_acceleration"):
        print("get data from acceleration")

        query_select_cmd = "SELECT * FROM acceleration where timestamp >= 
            '\%(time_min)s' and timestamp <= '\%(time_max)s' 
            and user_token = '\%(user_token)s'" \% filters
        cursor.execute(query_select_cmd)

        return {"data": cursor.fetchall()}
    
    elif(method_name == "get_location"):
        print("get data from location")

        query_select_cmd = "SELECT * FROM location where timestamp >= 
            '\%(time_min)s' and timestamp <= '\%(time_max)s' 
            and user_token = '\%(user_token)s'" \%filters
        cursor.execute(query_select_cmd)

        return {"data": cursor.fetchall()}
    else:
        return {"status": "error, route not found"}

if __name__ == "__main__":
    app.run(debug=True, port=5000)


def lambda_handler(event, context):
    return awsgi.response(app, event, context)

\end{python}



%%%%%%%%%%%%%%%%%%%%%%%%%%%%%%%



\section{Código para o Lambda da AWS de geração do relatório sobre o perfil do motorista}

\begin{python}

import awsgi
from flask import Flask, request

import requests
import json

import matplotlib.pyplot as plt
from svglib.svglib import svg2rlg

import smtplib
from email.mime.multipart import MIMEMultipart
from email.mime.text import MIMEText
from email.mime.application import MIMEApplication

from io import BytesIO
from reportlab.lib.pagesizes import letter
from reportlab.lib import utils
from reportlab.platypus import SimpleDocTemplate, Paragraph, Image
from reportlab.lib.styles import getSampleStyleSheet

from ride_parser import RealRideParser


app = Flask(__name__)

@app.route("/create_pdf", methods=["POST"])
def app_pdf():
    data = request.get_json()

    body_json_obd_info = {
        "method": "get_obd_info",
        "time_min" : data["time_min"],
        "time_max" : data["time_max"],
        "user_token" : data["user_token"]
    }

    body_json_acceleration = {
        "method": "get_acceleration",
        "time_min" : data["time_min"],
        "time_max" : data["time_max"],
        "user_token" : data["user_token"]
    }

    body_json_location = {
        "method": "get_location",
        "time_min" : data["time_min"],
        "time_max" : data["time_max"],
        "user_token" : data["user_token"]
    }

    data_obd_info = get_data_from_lambda(body_json_obd_info)
    data_acceleration = get_data_from_lambda(body_json_acceleration)
    data_location = get_data_from_lambda(body_json_location)


    params = {
        "user_id" : data["user_token"],
        "date_beg" : data["time_min"],
        "date_end" : data["time_max"]
    }

    ride_parser = RealRideParser(True, **params)
    metrics_map = ride_parser.generate_pdf_metrics()
    img = ride_parser.risk_table_graph

    fig = plt.figure(figsize=(4, 3))
    plt.plot([1,2,3,4])
    plt.ylabel('some numbers')

    create_PDF(img)
    return("pdf criado")



def get_data_from_lambda(body_json):
    url = "https://pntdpvkdsc.execute-api.us-east-1.amazonaws.com/default/app_data"

    headers = {
        'Content-Type': 'application/json',
    }

    response = requests.post(url, headers=headers, data=json.dumps(body_json))

    if response.status_code == 200:
        print("data successfully obtained from the server")
        return response
    else:
        print("failed to get data from the server")
        

def send_email(recipient_email, pdf_buffer):
    # Configurações do e-mail
    sender_email = "gabiru.bx@gmail.com"
    sender_password = "vbwiquhkjpjsmwdu"
    subject = "test email pdf"

    # Configuração do server SMTP do Gmail
    server_smtp = "smtp.gmail.com"
    port_smtp = 587

    # Criação do objeto MIMEMultipart
    message = MIMEMultipart()
    message["From"] = sender_email
    message["To"] = recipient_email
    message["Subject"] = subject

    # Adiciona o corpo do e-mail (opcional)
    corpo_email = "Corpo do teste pdf"
    message.attach(MIMEText(corpo_email, "plain"))

    # Adiciona o arquivo PDF como anexo
    attached_document = MIMEApplication(pdf_buffer.read(), _subtype="pdf")
    attached_document.add_header(
        "Content-Disposition", f"attachment; filename=anexo.pdf")
    message.attach(attached_document)

    # Inicia a conexão com o server SMTP
    print("starting server")
    server = smtplib.SMTP(server_smtp, port_smtp)
    server.starttls()

    # Efetua login no server
    server.login(sender_email, sender_password)

    print("sending email")
    # Envia o e-mail
    server.sendmail(sender_email, recipient_email, message.as_string())

    # Fecha a conexão com o server
    server.quit()

def create_PDF(metrics_map):
    # Create a PDF document
    pdf_buffer = BytesIO()
    doc = SimpleDocTemplate(pdf_buffer, pagesize=letter)

    # Define a style sheet
    styles = getSampleStyleSheet()

    # Create a list of paragraphs for the content
    content = []

    # Add title
    content.append(Paragraph("My Data Report", styles['Title']))

    # Add space
    content.append(Paragraph("<br/>", styles['BodyText']))

    # Salvar a figura em um buffer de bytes
    metrics_map.savefig(pdf_buffer, format='png')
    pdf_buffer.seek(0)

    drawing=svg2rlg(pdf_buffer)

    doc.build(content)

    # Reset the buffer position to the beginning before sending
    pdf_buffer.seek(0)

    send_email("gabriel.morghett@gmail.com", pdf_buffer)

if __name__ == "__main__":
    app.run(debug=True, port=5000)

def lambda_handler(event, context):
    return awsgi.response(app, event, context)

\end{python}

\section{Conversão de coordenadas geográficas em índices dos blocos no mapa}

\begin{python}
def convert_coord_to_chunk(self, coord_1, coord_2, max_coord_diff):
    coord_diff = coord_1 - coord_2
    chunk = (self.segmentation_rate * coord_diff) // max_coord_diff
    return chunk

    self.north = car_crimes_df["latitude"].max() + (0.5 / 111.11)
    self.south = car_crimes_df["latitude"].min() - (0.5 / 111.11)
    self.west = car_crimes_df["longitude"].max() + (0.5 / 111.11)
    self.east = car_crimes_df["longitude"].min() - (0.5 / 111.11)

    # maior distancia de norte a sul: 4378.4 km
    # maior distancia de leste a oeste: 4326.6 km
    self.segmentation_rate = 4400

    self.lat_diff = self.north - self.south
    self.long_diff = self.east - self.west

    
    car_crimes_df["chunk_i"] = car_crimes_df["latitude"]
        .apply(lambda x : 
        self.convert_coord_to_chunk(x, self.south, self.lat_diff))
    car_crimes_df["chunk_j"] = car_crimes_df["longitude"]
        .apply(lambda x : 
        self.convert_coord_to_chunk(x, self.west, self.long_diff))
\end{python}

