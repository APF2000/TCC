\chapter{Apêndices}

% Como juntar todas as subpartes do projeto é uma outra tarefaimportante de engenharia.

% A seguir é descrito como isso pode ser feito.

%%%%%%%%%%%%%%%%%%%

\section{Conversa com um profissional da Porto Seguro sobre seguros de carro}

%%%%%%%%%%%%%%%%%%%

\textbf{Pergunta 1:} 
Quanto aos locais por onde o motorista passa ao longo do dia, é mais importante saber o nome do bairro por onde ele passa ou se soubermos a rua é melhor pra sabermos o risco de assalto?

\textbf{Resposta 1:} 
A precificação do seguro já é baseada em três fatores: onde a pessoa mora, onde ela trabalha e o trajeto feito por ela para ir de um local ao outro.
Na nossa plataforma, chamada \textit{Corretor Online}, temos acesso a esses e outro dados do cliente, o que nos possibilita precificar o seguro de forma personalizada.

%%%%%%%%%%%%%%%%%%%

\textbf{Pergunta 2:} 
É importante a hora em que a pessoa passa em cada lugar? Ou dia da semana é um fator mais determinístico?

\textbf{Resposta 2:}
Não há correlação provada entre o dia da semana em que a pessoa dirige e a probabilidade de acidente de carro.

A hora do dia, no entanto, como pode ser intuitivo para muitos, determina a chance de o carro ser roubado. A maior parte dos crimes ocorre à noite.

%%%%%%%%%%%%%%%%%%%

% \textbf{Pergunta 3:} Momentos com mais trânsito podem ser um fator de risco para acidentes?


%%%%%%%%%%%%%%%%%%%

\textbf{Pergunta 3:} O sistema de precificação por uso é interessante pra seguros de motoristas de aplicativos? Quais parâmetros são mais relevantes para quem usa o carro prolongadamente?

\textbf{Resposta 3:}
Sim, com certeza seria interessante um sistema desse tipo.

Sobre os parâmetros exatos que seriam mais interessantes, preciso consultar a área estatística da empresa para ter mais certeza, mas certamente informações sobre como o motorista usa o carro (a quantidade de freadas bruscas ao longo do trajeto, por exemplo) e por onde passa com ele, conforme mencionei anteriormente, são fundamentais, na minha experiência.

%%%%%%%%%%%%%%%%%%%

\textbf{Pergunta 4:} 
Empresas podem se beneficiar da redução do seguro também? Levando-se em conta viagens feitas por funcionários, a trabalho

\textbf{Resposta 4:}
Hoje em dia nós seguramos empresas baseado puramente em quanto veículos têm na frota delas e quanto eles serão usados em média.

Se fosse possível diferenciar um motorista do outro, o cálculo do risco seria mais preciso, além de poder incentivar a boa condução.

Fizemos uma campanha com ideia parecida uma vez, chamada \textit{Trânsito Mais Gentil}. A ideia era diminuir o prêmio cobrado de cada segurado caso ele comprovasse, ao fim do ano, que cometeu poucas ou nenhuma infração no trânsito, dando-nos a quantidade de pontos recebidos na carteira.

%%%%%%%%%%%%%%%%%%%
    
\textbf{Pergunta 5:} 
Existe o interesse de ser feita uma precificação mais personalizada de seguros de carro, usando-se dados de estilo de direção de cada indivíduo em vez de estatísticas gerais?

\textbf{Resposta 5:}
Definitivamente sim, e acredito que essa seja a tendência do mercado no futuro próximo.

Existe uma plataforma que implementa algo parecido com o que vocês estão desenvolvendo, chama-se Azos\textsuperscript{[30]}.

Essa empresa precifica seguros de vida de forma dinâmica, levando em consideração não apenas estatísticas gerais sobre doenças que cada pessoa pode desenvolver, mas também e principalmente os hábitos do dia a dia de cada um e quais as novas condições de saúde do segurado a cada renovação de contrato.

%%%%%%%%%%%%%%%%%%%
    
\textbf{Pergunta 6:} 
Alguma consideração final?
    
\textbf{Resposta 6:} 
Quanto mais ajustado ao verdadeiro risco, melhor será o produto tanto para o cliente como para a seguradora, pois o preço mais justo para aquele serviço será o efetivamente cobrado.

Dito isso, vejo com muito bons olhos o projeto de vocês e contribuo com uma sugestão: algo tão personalizado assim poderia implementar um sistema equivalente ao \textit{Open Banking}, de que tanto se fala hoje em dia, uma vez que o histórico de cada cliente não é transferível entre empresas de seguro.

%%%%%%%%%%%%%%%%%%%


\section{Colunas da tabela do banco de dados}

\begin{python}
CREATE TABLE `tcc_main`.`info` (
  `id` INT NOT NULL AUTO_INCREMENT, 
  `timestamp` TIMESTAMP(3) NOT NULL,
  `user_token` VARCHAR(40),
  `name` VARCHAR(45) NOT NULL,
  `result` VARCHAR(45) NOT NULL,
  PRIMARY KEY (`id`),
  UNIQUE INDEX `id_UNIQUE` (`id` ASC) VISIBLE);

CREATE TABLE `acceleration` (
  `id` INT NOT NULL AUTO_INCREMENT,
  `timestamp` TIMESTAMP(3) NOT NULL,
  `user_token` VARCHAR(40),
  `acceleration_x` VARCHAR(10) NOT NULL,
  `acceleration_y` VARCHAR(10) NOT NULL,
  `acceleration_z` VARCHAR(10) NOT NULL,
  `gravity_x` VARCHAR(10) NOT NULL,
  `gravity_y` VARCHAR(10) NOT NULL,
  `gravity_z` VARCHAR(10) NOT NULL,
  PRIMARY KEY (`id`),
  UNIQUE INDEX `id_UNIQUE` (`id` ASC) VISIBLE);

CREATE TABLE `location` (
  `id` INT NOT NULL AUTO_INCREMENT,
  `timestamp` TIMESTAMP(3) NOT NULL,
  `user_token` VARCHAR(40),
  `latitude` VARCHAR(20) NOT NULL,
  `longitude` VARCHAR(20) NOT NULL,
  PRIMARY KEY (`id`),
  UNIQUE INDEX `id_UNIQUE` (`id` ASC) VISIBLE);

  \end{python}

% \section{Compra da placa ATMEGA}
% \section{Compra da placa ATMEGA}
% \section{Compra da placa ATMEGA}
% \section{Compra da placa ATMEGA}

