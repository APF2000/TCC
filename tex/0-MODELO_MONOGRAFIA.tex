\documentclass[ruledheader,noindentfirst,anapcustomindent,abntfigtabnum,tocpage=plain]{def_and_cls/abnt}
\usepackage{amsmath, amssymb, amsthm, verbatim, amsfonts, amstext}
%\usepackage[latin1]{inputenc}
\usepackage[brazilian]{babel}
\usepackage[utf8]{inputenc}
\usepackage[T1]{fontenc}
\usepackage{styles/dropping}
\usepackage{graphicx}
\usepackage[hang,small,bf]{caption}
\usepackage[abnt-etal-list=0,abnt-etal-text=it,abnt-and-type=&,abnt-emphasize=bf,abnt-full-initials=yes,alf,bibjustif]{styles/abntcite}
\usepackage{fancyhdr}
\usepackage{makeidx}
\usepackage[none]{hyphenat}
\usepackage{color}
\usepackage{subfig}
\usepackage{styles/algorithms}
\usepackage{algorithmic}
\usepackage{mdwlist}
\usepackage{bm}
\usepackage[titletoc,title]{appendix}
\usepackage{ltxtable}
\usepackage{longtable}
\usepackage{supertabular}
\usepackage{indentfirst}
\usepackage{color}
\usepackage{icomma}

\sloppy


%
%Tradução do pacote Algorithm para portugues
%
\renewcommand{\algorithmicrequire}{\textbf{Entrada:}}
\renewcommand{\algorithmicensure}{\textbf{Saída:}}
\renewcommand{\algorithmicend}{\textbf{fim}}
\renewcommand{\algorithmicif}{\textbf{se}}
\renewcommand{\algorithmicthen}{\textbf{então}}
\renewcommand{\algorithmicelse}{\textbf{senão}}
\renewcommand{\algorithmicelsif}{\algorithmicelse \, \algorithmicif}
\renewcommand{\algorithmicendif}{\algorithmicend \, \algorithmicif}
\renewcommand{\algorithmicfor}{\textbf{para}}
\renewcommand{\algorithmicforall}{\textbf{para todo}}
\renewcommand{\algorithmicdo}{\textbf{fazer}}
\renewcommand{\algorithmicendfor}{\algorithmicend \, \algorithmicfor}
\renewcommand{\algorithmicwhile}{\textbf{enquanto}}
\renewcommand{\algorithmicendwhile}{\algorithmicend \, \algorithmicwhile}
\renewcommand{\algorithmicloop}{\textbf{laço}}
\renewcommand{\algorithmicendloop}{\algorithmicend \, \algorithmicloop}
\renewcommand{\algorithmicrepeat}{\textbf{repetir}}
\renewcommand{\algorithmicuntil}{\textbf{até}}
\renewcommand{\algorithmiccomment}[1]{\{#1\}}
\renewcommand{\listalgorithmname}{Lista de Algoritmos}
\floatname{algorithm}{Algoritmo}
%%%%%%%%%%%%%%%%%%%%%%%%%%%%%%%%%%%%%%%%%%%%%%%%%%%%%%%%%%%%%%%%%%%%%%%%%%%%%%%%%%%

\makeindex

%%%% O arquivo modelosCAP.tex possui as definições para ciação do estilo de capítulo (fonte de título, barras horizontais, etc.)
% ele não gera texto de saída, é um arquivo de configuração somente
%
\input{0.1-modelosCAP}
%%%%%%%%%%%%%%%%%%%%%%%%%%%%%%%%%%%%%%%%%%%%%%%FIM DO PREAMBULO%%%%%%%%%%%%%%%%%%%%%%%%%%%%%%%%%%%%%%%%%%%%%%%%%%%%%%%%%%%%%%%%%%


\begin{document}

%%%%% IMPORTANTE: ALTERA O TEXTO ENTRE ARIAL E TIMES NEW ROMAN (ALTERNAR OS COMENTÁRIOS)
%
%%%%%%%%%%%%%%%%%%%%%PARA UTILIZAR ARIAL%%%%%%%%%%%%%%%%%%%%%%%
%
\fontfamily{phv}                    %fonte Arial
\renewcommand{\rmdefault}{phv}      %
%
%%%%%%%%%%%%%%%%%%%%%PARA UTILIZAR TIMES%%%%%%%%%%%%%%%%%%%%%%%
%
%\fontfamily{ptm}               %fonte Times
%\renewcommand{\rmdefault}{ptm} %
%
%%%%%%%%%%%%%%%%%%%%%%%%%%%%%%%%%%%%%%%%%%%%%%%%%%%%%%%%%%%%%%%

%%%%%%%%%%%%%Arquivos .tex com os elementos pré-textuais
%
\thispagestyle{empty}

\vfill
 \begin{center}
    {\large\bfseries ESCOLA POLITÉCNICA DA USP} \\
    \vspace*{1in}
    \begin{figure}[h]
     \centering
            \includegraphics[width=10cm]{figures/Logo_Poli.jpg}\\
     \end{figure}
    \vspace*{1in}
    \large\bfseries MONOGRAFIA DE SISTEMAS EMBARCADOS\\
    
    ESTABILIZADOR GIMBAL DE BAIXO CUSTO\\
    
    \vspace{1.5cm}
    ANA CLARA FORCELLI - 10773631\\
    ARTHUR PIRES DA FONSECA - 10773096\\
    BRENO LOSCHER ROCHA - 9784439\\
    \vfill
    \large\bfseries{ SÃO PAULO \\ 2022}
\end{center}

\normalsize
\begin{titlepage}
\vfill
\begin{center}
    \vspace{2cm}
    {\Large \textsc{MONOGRAFIA DE SISTEMAS EMBARCADOS\\
    ESTABILIZADOR DE CÂMERA DE BAIXO CUSTO}\\}
    \vspace{1cm}
    \hspace{.45\linewidth}
    \begin{minipage}{.50\linewidth}

            Um estabilizador 'gimbal' open-hardware e open-source utilizando materiais acessíveis

            \vspace{0.5 cm}

            Orientadores: \\
            Prof. Dr. Bruno C. Albertini\\
            Prof. Dr. Moacyr Martucci Junior\\
            Eng. Olimpio Rodrigues
    
    \end{minipage}

    \vspace{2cm}
    \vfill
    {\large SÃO PAULO\\ 2022}
\end{center}

\end{titlepage}
\include{3-folhadeaprovacao}
\chapter*{Dedicatória}

\noindent Dedico este trabalho à Hihi e aos meus avós por terem cuidado de mim, ao meu irmão (o Di), e aos meus pais, Alfredo e Márcia.

\noindent Dedico este trabalho ao meu amigo Mario Granziera e aos meus pais, Júlia e Antônio.
\chapter*{Agradecimentos}

\noindent Agradecimentos texto.\\

Agradecemos ao Guilherme Migliati https://www.linkedin.com/in/guimigli?utm_source=share&utm_campaign=share_via&utm_content=profile&utm_medium=android_app foram essenciais para superar obstáculos no desenvolvimento do aplicativo, oferecendo suporte técnico, que permitiu a compilação dele.

\\

Além disso, a contribuição significativa do Dr.-Ing. Philippe Jardin https://www.linkedin.com/in/dr-ing-philippe-jardin-964733a3?utm_source=share&utm_campaign=share_via&utm_content=profile&utm_medium=android_app , que disponibilizou o simulador de OBD quando o Arthur ainda estava na Alemanha, ampliou consideravelmente as capacidades de teste do sistema. 

\chapter*{Epígrafe}

\noindent Epígrafe texto.\\
\pagestyle{plain}%%%%% Utilizar ESTILO PLAIN AQUI%%%%%%%
\chapter*{Resumo}

\noindent Visando criar uma plataforma complementar ao motorista e seu carro, este trabalho usará sensores diversos ao redor de um automóvel para analisar o perfil de condução dessa pessoa.\\
A coleta será feita por uma porta OBD-II, presente em todos os veículos fabricados a partir de 2010 no Brasil.\\ 
Os dados, por sua vez, serão passados para a nuvem, para análise de perfil de cada condutor.\\ 
Cada condutor poderá consultar seus dados pessoais por um sistema com autenticação, exercitando, neste trabalho, a implementação da segurança, um aspecto fundamental para a proteção de dados.

\chapter*{Abstract}


\noindent Aiming to create a complementary platform to the driver and his car, this work will use various sensors around a car and from a smartphone, for example, to develop a data capturing platform. \\
The collection will be done through an OBD-II port, available in all vehicles manufactured since 2010 in Brazil. \\
The data, in turn, will be transferred to the cloud for profile analysis of each conductor. \\
Each driver will be able to consult their personal data through a system with authentication, which is an exercise of implementation of security, an aspect fundamental to data protection.

%%%Comandos para criação automática das listas
%
\tableofcontents
\listoffigures
\listoftables

%%%Comandos para criar outras listas não suportadas pelo pacote ABNTex%%%
%
\pretextualchapter{Lista de Símbolos}
\input{12-nomenclatura}
\newpage

\pretextualchapter{Lista de Abreviacoes}
\begin{basedescript}{\desclabelstyle{\pushlabel}\desclabelwidth{6em}}
\item[{IoT}] \textit{Internet of Things}
\item[{OBD-II}] \textit{On-board diagnostics}
\item[{MVP}] \textit{Minimum Viable Product}
\item[{IDE}] \textit{Integrated Development Environment}
\item[{JSON}] \textit{Java Script Object Notation}
\item[{API}] \textit{Application Programming Interface}
% \item[{fda}] Função de distribuição acumulada%
% \item[{EMQ}] Erro médio quadrático%
\end{basedescript}
\newpage
%%%%%%%%%%%%%%%%%%%%%%%%%%%%%%%%%%%%%%%%%%%%%%%%%%%%%%%%%%%%%%%%%%%%

%Capítulos passam a ter páginas numeradas
%
\pagestyle{fancy}

%resseta os contadores de capítulo e seção
%
\renewcommand{\chaptermark}[1]{\markboth{#1}{}}
\renewcommand{\sectionmark}[1]{\markright{\thesection\ #1}}

%%%%%%%%%%%%%%NÃO LEMBRO O QUE FAZ, APARENTEMENTE NADA, TESTAR DEPOIS
%\fancyhf{}%
%\fancyhead[RO,LE]{\large\slshape\thepage}%
%\fancyhead[CE]{\large\slshape\leftmark}%
%\fancyhead[CO]{\large\slshape\rightmark}%


%%% Outros arquivos .tex. É acoselhável utilizar vários arquivos, pelo menos um por capítulo
\chapter{Introdução}\label{CAP:introducao}

Este documento detalha a concepção e implementação de um sistema dedicado à aquisição e análise de dados relacionados à condução de um motorista.

A abordagem central consiste na extração de informações provenientes de sensores já presentes em veículos contemporâneos, acessando a porta OBD-II, que adota um padrão internacional para o diagnóstico de parâmetros internos de um automóvel\textsuperscript{[2]}.

Adicionalmente, são coletadas informações fornecidas por um dispositivo móvel, incluindo dados de geolocalização e acelerômetros, e essas informações são centralizadas em uma base de dados hospedada na AWS. 

Esta arquitetura permite uma análise posterior aprofundada desses dados, promovendo \textit{insights} valiosos sobre os padrões e comportamentos de direção.

\section{Motivação}

Buscando traçar o perfil de condução de um motorista e desenvolver uma categorização clara para os condutores, este projeto empenha-se na criação de uma infraestrutura dedicada à captação e análise de dados em veículos de uso pessoal.

Com base nos dados de aceleração, é possível classificar o estilo de direção, identificar comportamentos potencialmente perigosos no trânsito e antecipar a necessidade de manutenção do veículo. Ao integrar a análise dos dados ao sistema de perfilamento, não apenas contribui-se para a segurança do veículo, mas também viabiliza-se a personalização de \textit{feedbacks} e sugestões ao condutor, promovendo uma condução mais eficaz e segura.

Uma investigação preliminar sobre o tema revelou a existência de uma patente para um produto semelhante, registrada em 2013, que avalia o desempenho do motorista a partir de dados previamente coletados de parâmetros relevantes à condução do veículo\textsuperscript{[1]}.


\section{Objetivo}

Este trabalho procura criar uma plataforma de disponibilização e captura de dados em um carro, coletando as informações \textit{in loco} e apresentando estatísticas relevantes derivadas do que foi coletado.

Para isso, foi feito uso da infraestrutura já presente em carros atuais. Conforme pode ser visto na imagem \ref{fig:sensors_car}, alguns carros podem ter dezenas de sensores embarcados dentro de si.


\begin{figure}[hp]
    \centering
    
    \includegraphics[]{figures/sensores_carro.png}
    
    \caption{Veículos modernos estão equipados com dezenas de sensores\textsuperscript{[2]}.}
    
    \label{fig:sensors_car}
\end{figure}

Essas informações foram complementadas com o uso de um \textit{smartphone}, o qual disponibiliza outros tipos de dados graças a seus sensores embarcados e à sua conexão com os satélites de GPS.

A análise da aceleração desempenha um papel essencial na avaliação do comportamento do motorista em sistemas de coleta de dados. Ao monitorar os padrões de aceleração, é possível extrair informações valiosas sobre a condução, tais como a suavidade nas transições de velocidade, a resposta a alterações nas condições de tráfego e o nível de agressividade ao volante.

A aceleração é um parâmetro que revela a habilidade do motorista em manter uma condução estável e antecipar suas ações em situações específicas.
% Esse entendimento contribui para a compreensão do estilo de direção de cada condutor. 

% A maior parte da coleta de dados foi feita em um Hyundai Ix35, e dados estáticos foram tomadas de outros modelos também (Mercedes-Benz C200, Ford EcoSport e Volkswagen Saveiro) para conferência de parâmetros disponibilizados em comum por veículos de marcas diferentes.

Uma possível aplicação desse sistema encontra-se na área das empresas de seguro automotivo. Os dados e análises disponibilizados pelo aplicativo podem ser empregados na elaboração de políticas de precificação de seguros de veículos. 

Essa abordagem oferece a possibilidade de estabelecer um preço mais personalizado e justo para cada indivíduo, promovendo, assim, comportamentos mais responsáveis no trânsito. 

% Essa iniciativa não apenas contribui para a equidade na precificação, mas também estimula práticas de condução mais seguras e conscientes.

 
\section{Justificativa}
% ROMEO e PIRES [GIT]
% Pq o trabalho é importante?
No ano de 2021, 11.647 pessoas perderam a vida em acidentes de trânsito no Brasil\textsuperscript{[6]}. Além disso, a nível global, a média anual de fatalidades no trânsito atinge 1,35 milhão de pessoas\textsuperscript{[7]}, um número comparável às mortes por Covid-19 até abril de 2022\textsuperscript{[8]}.

Diante desses dados alarmantes, a importância deste projeto torna-se evidente, uma vez que estabelece métricas significativas para a avaliação do comportamento dos motoristas. Essas métricas não apenas contribuem para a prevenção de acidentes, mas, sobretudo, para a preservação da vida humana. O projeto emerge como uma ferramenta valiosa na promoção da segurança nas ruas e na busca por soluções que possam mitigar essas estatísticas trágicas.

Os veículos autônomos estão rapidamente conquistando popularidade, de forma que eles tornarem-se a maioria nas estradas é uma realidade iminente. Prevê-se, de fato, que carros com pelo menos nível 4 de automação tornem-se mais comuns a partir de 2025\textsuperscript{[9]}.

Dessa forma, embora a influência humana possa perder relevância em acidentes do futuro, torna-se imperativo estabelecer métricas para avaliar a condução de veículos autônomos. 

Essa abordagem é essencial para assegurar a segurança e eficiência desses automóveis, promovendo um padrão consistente de avaliação que contribua para a confiança generalizada no crescente uso de tecnologias autônomas.

Este projeto tem, portanto, extrema relevância, pois é uma possível ferramenta para reduzir as fatalidades no trânsito, independentemente de serem causadas por condutores humanos ou por veículos autônomos.

\section{Organização do trabalho}
Os próximos capítulos explicitarão como o trabalho foi planejado a partir de seus requisitos e das tecnologias envolvidas.

O projeto consistiu em integrar a porta OBD-II de qualquer carro com a plataforma de armazenamento de dados que foi criada.

Para isso, o Capítulo 2 faz uma breve apresentação dos conceitos usados neste projeto. O Capítulo 3, por sua vez, traz as etapas de desenvolvimento do trabalho e o Capítulo 4 mostra quais requisitos foram definidos para o sistema. Além disso, o desenvolvimento inicial do trabalho é descrito no Capítulo 5. Por último, o Capítulo 6 traz as considerações finais com uma breve discussão das conclusões do trabalho, assim como contribuições e sugestões para a continuidade dele.
\chapter{Aspectos conceituais}
\label{CAP2}

A seguir serão explicados os conceitos fundamentais que possibilitam a execução deste trabalho. 

% apresentar conceitos empregados e revisao da literatura (parte toerica do trab)

\section{O padrão OBD-II}

Criado na década de 1990 para gerar controle sobre as emissões de gás carbônico dos carros\textsuperscript{[5]}, hoje define um protocolo padronizado para comunicar parâmetros internos do veículo.
    
O padrão OBD-II foi uma extensão do padrão OBD-I, uniformizando esse tipo de conector em casos em geral, começando a ser adotado no Brasil a partir de 2010\textsuperscript{[4]}.

A comunicação é feita através de uma conexão física que normalmente pode ser encontrada abaixo do volante do motorista\textsuperscript{[5]}, conforme indicado pela figura \ref{fig:obd2_conn}.

Nesse padrão de conexão e comunicação é definida uma interface por onde parâmetros internos a um carro podem ser monitorados.

As mensagens definidas pelo OBD-II têm cada uma um PID (parameter ID). O conjunto comum de PIDs de serviços que podem ser solicitados pela porta OBD-II e para que servem pode ser visto na tabela \ref{Tb:tab1}\textsuperscript{[3]}.

\begin{table}[]
% \begin{adjustbox}{width=\textwidth}
\begin{tabular}{cc}
\rowcolor[HTML]{656565} 
{\color[HTML]{FFFFFF} Service / Mode (hex)} & {\color[HTML]{FFFFFF} Description}                                                                    \\
01                                          & Show current data - I/M Monitors and Live Data                                                        \\
02                                          & Show Freeze Frame (FF) Data                                                                           \\
03                                          & Show Stored Diagnostic Trouble Codes                                                                  \\
04                                          & Clear/Erase Diagnostic Trouble Codes and stored values                                                \\
05                                          & Test results, oxygen sensor monitoring (non CAN only)                                                 \\
06                                           & Test results, other component/system monitoring (Test results, oxygen sensor monitoring for CAN only) \\
07                                           & Show pending Diagnostic Trouble Codes (detected during current or last driving cycle)                 \\
08                                           & Control operation of on-board component/system (EVAP)                                                 \\
09                                          & Request Vehicle Information (VIN)                                                                     \\
0A                                          & Permanent Diagnostic Trouble Codes (DTCs) (Cleared DTCs)                                             
\end{tabular}
\end{table}

É importante notar que os serviços do padrão OBD-II comuns a todos os carros sempre começam com o dígito zero (hexadecimal), por isso, serviços específicos de cada fabricante devem começam a partir do código 0x10.

\begin{figure}[hp]
    \centering
    
    \includegraphics[]{figures/localizacao_obd2.png}
    
    \caption{Posição da porta OBD-II em um carro\textsuperscript{[5]}}
    
    \label{fig:obd2_conn}
\end{figure}

Os dados que podem ser coletados de qualquer carro são explicitados na tabela \ref{Tb:tab2}\textsuperscript{[3]}.

\begin{table}[h]
% \begin{adjustbox}{width=\textwidth}
    \caption{PIDs do OBD-II e seus valores de referência.}
    \label{Tb:tab2}
    \centering
    
    \begin{adjustbox}{width=\textwidth}
    
        \begin{tabular}{ccc p{6cm} ccc}
            \rowcolor[HTML]{C0C0C0} 
            
            {\color[HTML]{FFFFFF} \textbf{PIDs(hex)}} & {\color[HTML]{FFFFFF} \textbf{PID(Dec)}} & {\color[HTML]{FFFFFF} \textbf{Data bytes returned}} & {\color[HTML]{FFFFFF} \textbf{Description}} & {\color[HTML]{FFFFFF} \textbf{Min value}} & {\color[HTML]{FFFFFF} \textbf{Max value}} & {\color[HTML]{FFFFFF} \textbf{Units}} \\
            
            04 & 4 & 1           & Calculated engine load                      & 0 & 100 & \% \\
            0C & 12 & 2           & Engine speed & 0 & 16,383.75 & rpm \\
            0D & 13 & 1           & Vehicle speed & 0 & 255 & km/h \\
            0F & 15 & 1           & Intake air temperature                      & -40 & 215 & °C \\
            1F & 31 & 2           & Run time since engine start                 & 0 & 65,535 & seconds \\
            33 & 51 & 1           & Absolute Barometric Pressure                & 0 & 255 & kPa \\
            46 & 70 & 1           & Ambient air temperature                     & -40 & 215 & °C \\
            47 & 71 & 1           & Absolute throttle position B                & 0 & 100 & \% \\
            49 & 73 & 1           & Accelerator pedal position D                & 0 & 100 & \% \\
            51 & 81 & 1           & Fuel Type & - & - & - \\
            52 & 82 & 1           & Ethanol fuel \% & 0 & 100 & \% \\
            5A & 90 & 1           & Relative accelerator pedal position         & 0 & 100 & \% \\
            5B & 91 & 1           & Hybrid battery pack remaining life          & 0 & 100 & \% \\
            5C & 92 & 1           & Engine oil temperature                      & -40 & 210 & °C \\
            5D & 93 & 2           & Fuel injection timing & -210.00 & 301992 & ° \\
            5E & 94 & 2           & Engine fuel rate & 0 & 3212.75 & L/h \\
            61 & 97 & 1           & Driver's demand engine - percent torque     & -125 & 130 & \% \\
            62 & 98 & 1           & Actual engine - percent torque              & -125 & 130 & \% \\
            63 & 99 & 2           & Engine reference torque                     & 0 & 65,535 & Nm \\
            64 & 100 & 5           & Engine percent torque data                  & -125 & 130 & \% \\
            70 & 112 & 10          & Boost pressure control                      & - & - & - \\
            83 & 131 & 9           & NOx sensor & - & - & - \\
            8E & 142 & 1           & Engine Friction - Percent Torque            & -125 & 130 & \
            %     https://pt.overleaf.com/project/625c13a8f632581c97970222 
        \end{tabular}
    \end{adjustbox}
\end{table}

% \noindent

% \begin{tblr}[caption = {Modelos estatísticos e suas relações.}]{
%       hlines,
%       vlines,
%       row{1} = {bg=gray,fg=white},
%       rows = {halign=c},
%       columns = {halign=c},
%     %   column{4} = {Q[3cm]},
%     %   column{2} = {Q[1cm] XXX}
%       colspec = {Q[1.2cm, halign=c] XXX },
%   } 
%     \label{Tb:tab1}
%     \centering
  
%   \rowcolor[HTML]{C0C0C0} 
            
%             {\color[HTML]{FFFFFF} \textbf{PIDs (hex)}} & {\color[HTML]{FFFFFF} \textbf{PID (Dec)}} & {\color[HTML]{FFFFFF} \textbf{Data bytes returned}} & {\color[HTML]{FFFFFF} \textbf{Description}} & {\color[HTML]{FFFFFF} \textbf{Min value}} & {\color[HTML]{FFFFFF} \textbf{Max value}} & {\color[HTML]{FFFFFF} \textbf{Units}} \\
  
%               04 & 4 & 1           & Calculated engine load                      & 0 & 100 & \% \\
%             0C & 12 & 2           & Engine speed & 0 & 16,383.75 & rpm \\
%             0D & 13 & 1           & Vehicle speed & 0 & 255 & km/h \\
%             0F & 15 & 1           & Intake air temperature                      & -40 & 215 & °C \\
%             1F & 31 & 2           & Run time since engine start                 & 0 & 65,535 & seconds \\
%             33 & 51 & 1           & Absolute Barometric Pressure                & 0 & 255 & kPa \\
%             46 & 70 & 1           & Ambient air temperature                     & -40 & 215 & °C \\
%             47 & 71 & 1           & Absolute throttle position B                & 0 & 100 & \% \\
%             49 & 73 & 1           & Accelerator pedal position D                & 0 & 100 & \% \\
%             51 & 81 & 1           & Fuel Type & - & - & - \\
%             52 & 82 & 1           & Ethanol fuel \% & 0 & 100 & \% \\
%             5A & 90 & 1           & Relative accelerator pedal position         & 0 & 100 & \% \\
%             5B & 91 & 1           & Hybrid battery pack remaining life          & 0 & 100 & \% \\
%             5C & 92 & 1           & Engine oil temperature                      & -40 & 210 & °C \\
%             5D & 93 & 2           & Fuel injection timing & -210.00 & 301992 & ° \\
%             5E & 94 & 2           & Engine fuel rate & 0 & 3212.75 & L/h \\
%             61 & 97 & 1           & Driver's demand engine - percent torque     & -125 & 130 & \% \\
%             62 & 98 & 1           & Actual engine - percent torque              & -125 & 130 & \% \\
%             63 & 99 & 2           & Engine reference torque                     & 0 & 65,535 & Nm \\
%             64 & 100 & 5           & Engine percent torque data                  & -125 & 130 & \% \\
%             70 & 112 & 10          & Boost pressure control                      & - & - & - \\
%             83 & 131 & 9           & NOx sensor & - & - & - \\
%             8E & 142 & 1           & Engine Friction - Percent Torque            & -125 & 130 & \
% \end{tblr}

Interessante notar que, embora não esteja representado na tabela, os PIDs 0x00, 0x20, 0x40, etc, ou seja, a cada 32 valores de PID, existe um parâmetro apenas para indicar quais entre os PIDs seguintes são fornecidos por aquele carro.

Esses PIDs de marcação, por assim dizer, utilizam 4 bytes para comunicar quais dos próximos PIDs estão disponíveis, utilizando cada bit como uma flag binária. 

Na figura \ref{fig:bitwise_obd2} é possível visualizar como esse processo é feito, usando como exemplo o valor 0xBE1FA813 para representar os 4 bytes oferecidos\textsuperscript{[3]}.

\begin{figure}[hp]
    \centering
    
    \includegraphics[scale=0.7]{figures/tabela_dados_disponiveis.png}
    
    \caption{Divisão bit a bit da mensagem OBD-II informando os serviços disponíveis\textsuperscript{[3]}}
    
    \label{fig:bitwise_obd2}
\end{figure}

A coleta de fato dos dados relevantes será feita por uma aplicação desenvolvida \textit{a priori} que já é capaz de comunicar-se corretamente com a interface OBD.

Essa aplicação acelerará a criação da infraestrutura proposta por este trabalho, uma vez que pula o primeiro passo da estratégia \textit{bottom-up} que deverá permear o projeto.

\section{Transmissão de dados}

A implementação da transmissão Bluetooth neste projeto de rastreamento de veículos oferece uma solução eficaz para a comunicação entre o sistema de monitoramento e dispositivos móveis. 

Ao utilizar a tecnologia Bluetooth, os dados de telemetria e rastreamento podem ser transmitidos de forma sem fio e em tempo real para dispositivos equipados com Bluetooth, como smartphones ou tablets. Essa abordagem sem fio proporciona uma conectividade mais flexível, eliminando a necessidade de cabos físicos e simplificando a integração do sistema. A baixa energia do Bluetooth permite uma transmissão eficiente de dados, minimizando o impacto no consumo de bateria dos dispositivos móveis.

Além disso, a transmissão Bluetooth possibilita a interação direta entre o sistema de rastreamento e os usuários, permitindo a visualização em tempo real de informações de condução, alertas e feedbacks personalizados por meio de aplicativos dedicados. Essa integração sem fio não apenas aprimora a experiência do usuário, mas também amplia as possibilidades de interatividade e controle, tornando o sistema de rastreamento de veículos mais acessível e fácil de usar.

\section{Armazenamento em nuvem}
Provedoras como a Amazon e a Azure (Microsoft) oferecem serviços de nuvem que podem ser usados para este projeto \textsuperscript{[10, 11]}.Neste projeto, optou-se pelo serviço da AWS, uma vez que um dos integrantes do grupo já tinha familiaridade com o serviço.

A integração desse sistema com o Amazon RDS (Relational Database Service) da AWS oferece uma solução eficiente e escalável para o armazenamento de dados. Utilizando o RDS, os dados coletados, como informações de localização, velocidade, e padrões de condução, podem ser armazenados em bancos de dados relacionais, proporcionando alta disponibilidade, segurança e desempenho otimizado. 

A flexibilidade do Amazon RDS permite a escolha de diferentes motores de banco de dados, como MySQL, PostgreSQL, ou SQL Server, adequando-se às necessidades específicas do sistema.

A capacidade de escalabilidade automática do RDS facilita a gestão de grandes volumes de dados, garantindo que o sistema de rastreamento possa crescer conforme necessário. Além disso, os recursos de backup automático e a conformidade com padrões de segurança robustos da AWS asseguram a integridade e confidencialidade dos dados armazenados, proporcionando uma base sólida para análises futuras e insights valiosos sobre o comportamento de condução.

Os usuários poderão transmitir informações para a nuvem a qualquer momento, desde que seus carros estejam ligados e conectados à internet. O volume de dados recebidos no sistema é, portanto, variável e por isso fez sentido que o serviço de nuvem contratado siga o modelo \textit{on demand}.
    \chapter{Microcontrolador}
\label{CAP3}


Este capítulo tem como objetivo 


\section{aaaaaaaaaaa}\label{Sub:equa}
ATmega328P
Motivação:
Fácil de implementar, ajuda na ideia de open-hardware
Memória programável já dentro do sistema (alguns ATs não têm)
Não é caro (maior parte do custo será nos componentes mecânicos)
Consumo de energia baixo (22mW) em relação à potência proposta (de 1W)

\chapter{Análise de mercado}

\label{CAP4}


Este capítulo tem como objetivo 


\section{bbbbbbbbbb}\label{Sub:equa}

Linha Zhiyun Smooth 
Só serve para celulares, o nosso estabilizador servirá para câmeras semi-profissionais  também
Por serem feitos para câmeras profissionais, precisam de mais robustez, por isso, custam mais caro:
Linha Gimbal DJI RS
Valor acima de 2500 reais
Linha S Leeremi
Valor acima de 1000 reais


\chapter{Desenvolvimento do trabalho}]

\label{CAP5}


Este capítulo tem como objetivo 


\section{Tecnologias utilizadas???? (nao é aqui) talvez Estrutura}
interface com usuario, banco de dados, conexão obd2

\section{Projeto e implementação}
decisoes feitas durante o trab

\section{Testes e avaliação}
descrever o plano de testes do sistema: testes de software, modulo, integração e validação

Using \texttt{biblatex} you can display a bibliography divided into sections, depending on citation type. 
Let's cite! Einstein's journal paper \cite{einstein} and Dirac's book \cite{dirac} are physics-related items. 
Next, \textit{The \LaTeX\ Companion} book \cite{latexcompanion}, Donald Knuth's website \cite{knuthwebsite}, \textit{The Comprehensive Tex Archive Network} (CTAN) \cite{ctan} are \LaTeX-related items; but the others, Donald Knuth's items, \cite{knuth-fa,knuth-acp} are dedicated to programming. 


\chapter{Considerações finais}]
\label{CAP6}


Este capítulo tem como objetivo 


\section{Conclusões do projeto de formatura}
"foi bom, tivemos que mudar isso e isso e isso"

\section{Contribuições}
qual foi a contrib da equipe

\section{Perspectivas de continuidade}
o que pode ser feito depois do tcc?
\include{20-cap7}
\include{tex/21-apêndices}
\include{tex/22-anexos}
\chapter{Referências}

[1] “US20130052614A1 - Driver Performance Metric.” Google Patents, Google, disponível em <patents.google.com/patent/US20130052614>. Acesso em 30 de janeiro de 2022

[2] Paulo Campo Grande. “Novas Tecnologias: Carros Atuais Têm Até 100 Sensores a Bordo.” Quatro Rodas, 12 de junho de 2018, disponível em <quatrorodas.abril.com.br/noticias/novas-tecnologias-carros-atuais-tem-ate-100-sensores-a-bordo/>. Acesso em 20 de fevereiro de 2022

[3] “OBD-II PIDs.” Wikipedia, Wikimedia Foundation, 17 Mar. 2022, disponível em <en.wikipedia.org/wiki/OBD-II_PIDs>. Acesso em 20 de março de 2022

[4] Sobre o autor Nina Finco Formada em jornalismo pela UMESP-SP e especialista em mídia. “OBD: o Que é e Para Que Serve o Protocolo OBD2?” Blog Da Cobli, 9 Sept. 2021, disponível em <www.cobli.co/blog/o-que-e-protocolo-obd2/>. Acesso em 20 de março de 2022

[5] “What Is OBDII? History of on-Board Diagnostics.” Geotab, disponível em <www.geotab.com/blog/obd-ii/>. Acessado em 20 de março de 2022

[6] https://www.sbtnews.com.br/noticia/brasil/194388-no-brasil--cerca-de-32-pessoas-morrem-por-dia-em-acidentes-de-transito#:~:text=Em%202021%2C%20foram%2011.647%20mortes,incidentes%20por%20hora%20no%20Brasil.

[7] https://news.un.org/pt/story/2021/11/1771092

[8] https://ourworldindata.org/grapher/covid-deaths-income, consultado em 18 de abril de 2022

[9] https://www.jaguarbrasil.com.br/news/em-quanto-tempo-os-carros-autonomos-serao-o-novo-padrao.html#:~:text=A%20ind%C3%BAstria%20de%20pesquisa%20IHS,pode%20demorar%20um%20pouco%20mais.




%%%% Estilo de citação ABNT e arquivo de bibitens (mybibliography.bib)
\bibliographystyle{abnt-alf}
\bibliography{mybibliography}

\apendice
\include{appendices}


\end{document}