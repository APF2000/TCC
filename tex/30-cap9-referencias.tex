\chapter{Referências}

\urlstyle{same} % mantém fonte das letras que compõem url

[1] \textbf{US20130052614A1 - Driver Performance Metric}. Google Patents, 2012. Disponível em <\url{patents.google.com/patent/US20130052614}>. Acesso em 30 de jan. de 2022.

[2] CAMPO GRANDE, Paulo. \textbf{Novas Tecnologias: Carros Atuais Têm Até 100 Sensores a Bordo}. Revista Quatro Rodas, Brasil, 12 de jun. de 2018. Disponível em <\url{quatrorodas.abril.com.br/noticias/novas-tecnologias-carros-atuais-tem-ate-100-sensores-a-bordo/}>. Acesso em 20 de fev. de 2022.

% 17 de mar de 
[3] WIKIMEDIA Foundation. \textbf{OBD-II PIDs}. Wikipedia, 2022. Disponível em <\url{en.wikipedia.org/wiki/OBD-II_PIDs}>. Acesso em 20 de mar. de 2022

% 9 de set. de 
[4] FINCO, Nina. \textbf{OBD}: o Que é e Para Que Serve o Protocolo OBD2?. Cobli Blog, 2021. Disponível em <\url{www.cobli.co/blog/o-que-e-protocolo-obd2/}>. Acesso em 20 de mar. de 2022.

% 25 de nov. de
[5] BARRETO, Victor. \textbf{What Is OBDII?} History of on-Board Diagnostics. Geotab, 2020 Disponível em <\url{www.geotab.com/blog/obd-ii/}>. Acesso em 20 de mar. de 2022.

[6] SBT News. \textbf{No Brasil, cerca de 32 pessoas morrem por dia em acidentes de trânsito}. SBT, Brasil, 22 de jan. de 2022. Disponível em <\url{www.sbtnews.com.br/noticia/brasil/194388-no-brasil--cerca-de-32-pessoas-morrem-por-dia-em-acidentes-de-transito#:~:text=Em 2021, foram 11.647 mortes,incidentes por hora no Brasil.}>. Acesso em 23 de abr. de 2022.

[7] \textbf{ACIDENTES De Trânsito São a Maior Causa De Morte De Pessoas De 5 a 29 Anos}. ONU News. Nações Unidas, 21 de nov. de 2021. Disponível em <\url{news.un.org/pt/story/2021/11/1771092}>. Acesso em 18 de abr. de 2022.

[8] \textbf{TOTAL Confirmed COVID-19 Deaths}. Our World in Data, 2022. Disponível em <\url{ourworldindata.org/grapher/covid-deaths-income}>
. Acesso em 18 de abr. de 2022.

[9] \textbf{Em Quanto Tempo Os Carros Autônomos Serão o Novo 'Padrão'?}. Jaguar Brasil. Disponível em <\url{www.jaguarbrasil.com.br/news/em-quanto-tempo-os-carros-autonomos-serao-o-novo-padrao.html#:~:text=A indústria de pesquisa IHS,pode demorar um pouco mais}>. Acesso em 23 de abr. de 2022.


[10] \textbf{PREÇO sob demanda do Amazon EC2}. Amazon. Disponível em <\url{aws.amazon.com/pt/ec2/pricing/on-demand/}>. Acesso em 23 de abr. de 2022.


[11] \textbf{CALCULADORA de preço}. Microsoft Azure. Disponível em <\url{azure.microsoft.com/de-de/pricing/calculator/}>. Acesso em 23 de abr. de 2022.

[12] PIRES. \textbf{Android OBD-II Reader Application That Uses Pure OBD-II PID's Java API}. GitHub. Disponível em <\url{github.com/pires/android-obd-reader}>. Acesso em 23 de abr. de 2022.

[13] PAGE, Vanessa. \textbf{Waze: The Pros and Cons}. Investopedia, 12 de jan de 2022. Disponível em <\url{www.investopedia.com/articles/investing/060415/pros-cons-waze.asp#:~:text=Waze uses data from app,that could slow down drivers}>. Acesso em 23 de abr. de 2022.


[14] \textbf{Scanner Automotivo Conector Obd2 Elm327 Bluetooth}. Mercado Livre, <\url{produto.mercadolivre.com.br/MLB-2147325216-scanner-automotivo-conector-obd2-elm327-bluetooth-_JM#position=1&search_layout=grid&type=pad&tracking_id=509523dc-c530-481c-954c-0ecbe9c6a94c&is_advertising=true&ad_domain=VQCATCORE_LST&ad_position=1&ad_click_id=ODlhMjFlMzctNjQ5Ni00MTg1LWIzYWYtZjk3Y2U5MjlmNjNm}>. Acesso em 23 de abr. de 2022.

[15] SAIPRASERT, Chalermpol. et al. \textbf{Driver Behaviour Profiling Using Smartphone Sensory Data in a V2I Environment}. 2014 International Conference on Connected Vehicles and Expo (ICCVE), 2014. Disponível em <\url{https://ieeexplore.ieee.org/abstract/document/7297609}>. Acesso em 24 de abr. de 2022.

[16] \textbf{Imagem MySql e AWS RDS}. Disponível em <\url{https://medium.com/99dotco/our-mysql-rds-upgrade-journey-cutting-down-downtime-by-11200-and-lessons-learned-1fa828e6009c}>. Acesso em 20 de nov. de 2023.

[17] \textbf{Imagem Python e Folium}. Disponível em <\url{https://www.google.com/url?sa=i&url=https%3A%2F%2Fm.youtube.com%2Fwatch%3Fv%3D4RnU5qKTfYY&psig=AOvVaw05zKbpT4lYD0v7v8M3hIZe&ust=1700594334592000&source=images&cd=vfe&opi=89978449&ved=0CA8QjRxqFwoTCMCfq-Kl04IDFQAAAAAdAAAAABAD}>. Acesso em 20 de nov. de 2023.

[18] \textbf{Simulador Carla}. Disponível em <\url{https://carla.org/}>

[19] \textbf{Site Rota2030}. Disponível em <\url{https://www.rota2030.com.br/}>

[20] \textbf{Dosovitskiy, A., Ros, G., Codevilla, F., Lopez, A., Koltun, V. (2017, October)}. CARLA: An open urban driving simulator. In Conference on robot learning (pp. 1-16). PMLR.

[21] \textbf{Simulador OBD-II}. Disponivel em <\url{https://freematics.com/store/index.php?route=product/product&product_id=71}>

[22] \textbf{Ferreira Júnior, Jair da Silva}.
Análise de Perfil de Motoristas : Uma Investigação com Diferentes Sensores e Técnicas de
Aprendizado de Máquina / Jair da Silva Ferreira Júnior. — 2018
101 f. : il. color.

[23] \textbf{Shibata, Danilo Jun; Soares, Leandro Donizetti}. Ánalise de Direção de motoristas Utilizando o Protocolo OBD2 e sensores Embarcados.

[24] \textbf{Bethge, Johanna, et al}."Model Predictive Control with Gaussian-Process-Supported Dynamical Constraints for Autonomous Vehicles." arXiv preprint arXiv:2303.04725 (2023).

[25] \textbf{Ferreira J Júnior, Carvalho E, Ferreira BV, de Souza C, Suhara Y, et al. (2017)}. Driver behavior profiling: An investigation with different smartphone sensors and machine learning. PLOS ONE 12(4): e0174959.

[26] \textbf{Júnior, J. Ferreira, and Gustavo Pessin}. "Análise de perfil de motoristas: Detecção de eventos por meio de smartphones e aprendizado de máquina." Anais do WOCCES 2016 Workshop de Comunicação em Sistemas Embarcados Críticos. 2016.